This is Tiny protocol implementation for microcontrollers (Arduino, Stellaris).\hypertarget{arduino_arduino_tiny}{}\section{Simple Tiny Protocol examples}\label{arduino_arduino_tiny}
Simple Tiny Protocol examples section is applicable only when working with S\+I\+M\+P\+L\+E\+\_\+\+A\+PI and A\+D\+V\+A\+N\+C\+E\+D\+\_\+\+A\+PI. If you want to work with Tiny Half Duplex protocol, please refere to \hyperlink{arduino_arduino_tiny_hd}{Half Duplex Tiny Protocol examples} section.\hypertarget{arduino_arduino_tiny_init}{}\subsection{Initialization}\label{arduino_arduino_tiny_init}

\begin{DoxyCode}
Tiny::Proto proto;

\textcolor{keywordtype}{void} setup()
\{
    proto.beginToSerial();
\}
\end{DoxyCode}
\hypertarget{arduino_arduino_tiny_send}{}\subsection{Sending/\+Receiving data over protocol}\label{arduino_arduino_tiny_send}
\hypertarget{arduino_arduino_tiny_send_receive1}{}\subsubsection{Variant 1}\label{arduino_arduino_tiny_send_receive1}
First variant\+: without using any helpers to work with data 
\begin{DoxyCode}
Tiny::Proto proto;

\textcolor{comment}{/* The buffer must be defined globally */}
uint8\_t g\_buffer[16];

\textcolor{keywordtype}{void} loop()
\{
    \textcolor{keywordflow}{if} (needToSend)
    \{
        \textcolor{comment}{/* define buffer, you want to use */}
        uint8\_t buffer[16];

        \textcolor{comment}{/* Prepare data you want to send here */}
        buffer[0] = 10;
        buffer[1] = 20;

        \textcolor{comment}{/* Send 2 bytes to remote side */}
        proto.write( buffer, 2 );
    \}
    \textcolor{keywordtype}{int} length;
    length = proto.read( g\_buffer, \textcolor{keyword}{sizeof}(g\_buffer), TINY\_FLAG\_NO\_WAIT );
    \textcolor{keywordflow}{if} ( length > 0 )
    \{
        \textcolor{comment}{/* Parse your data received from remote side here. */}
    \}
\}
\end{DoxyCode}
\hypertarget{arduino_arduino_tiny_send_receive2}{}\subsubsection{Variant 2}\label{arduino_arduino_tiny_send_receive2}
Second variant\+: with using special helper to pack the data being sent 
\begin{DoxyCode}
Tiny::Proto proto;

\textcolor{comment}{/* The buffer must be defined globally */}
uint8\_t g\_buffer[16];
Tiny::Packet packet(g\_buffer, \textcolor{keyword}{sizeof}(g\_buffer));

\textcolor{keywordtype}{void} loop()
\{
    \textcolor{keywordflow}{if} (needToSend)
    \{
        \textcolor{comment}{/* define buffer, you want to use */}
        uint8\_t buffer[16];
        \textcolor{comment}{/* Create helper object to simplify packing of data being sent */}
        Tiny::Packet packet(buffer, \textcolor{keyword}{sizeof}(buffer));

        \textcolor{comment}{/* Pack data you want to send here */}
        packet.put( \textcolor{stringliteral}{"message"} );

        \textcolor{comment}{/* Send message to remote side */}
        proto.write( packet );
    \}
    \textcolor{keywordtype}{int} length;
    length = proto.read( packet, TINY\_FLAG\_NO\_WAIT );
    \textcolor{keywordflow}{if} ( length > 0 )
    \{
        \textcolor{comment}{/* Parse your data received from remote side here. */}
        \textcolor{comment}{/* For example, read sent ealier in example above "message" */}
        \textcolor{keywordtype}{char} * str = packet.getString();
    \}
\}
\end{DoxyCode}
\hypertarget{arduino_arduino_tiny_close}{}\subsection{Stopping communication}\label{arduino_arduino_tiny_close}

\begin{DoxyCode}
Tiny::Proto proto;

\textcolor{keywordtype}{void} loop()
\{
    ...
    \textcolor{keywordflow}{if} ( needToStop )
    \{
        proto.end();
    \}
\}
\end{DoxyCode}
\hypertarget{arduino_arduino_tiny_hd}{}\section{Half Duplex Tiny Protocol examples}\label{arduino_arduino_tiny_hd}
Half Duplex Tiny Protocol examples section is applicable when working with H\+A\+L\+F\+\_\+\+D\+U\+P\+L\+E\+X\+\_\+\+A\+PI, S\+I\+M\+P\+L\+E\+\_\+\+A\+PI and A\+D\+V\+A\+N\+C\+E\+D\+\_\+\+A\+PI.\hypertarget{arduino_arduino_tiny_hd_init}{}\subsection{Initialization}\label{arduino_arduino_tiny_hd_init}

\begin{DoxyCode}
uint8\_t g\_buffer[64];

\textcolor{keywordtype}{void} onReceiveData(uint8\_t *buffer, \textcolor{keywordtype}{int} len)
\{
\}

Tiny::ProtoHd proto(g\_buffer, \textcolor{keyword}{sizeof}(g\_buffer), onReceiveData);

\textcolor{keywordtype}{void} setup()
\{
    proto.beginToSerial();
\}
\end{DoxyCode}
\hypertarget{arduino_arduino_tiny_hd_send_receive}{}\subsection{Sending/\+Receiving data over protocol}\label{arduino_arduino_tiny_hd_send_receive}
\hypertarget{arduino_arduino_tiny_hd_send_receive1}{}\subsubsection{Variant 1}\label{arduino_arduino_tiny_hd_send_receive1}
First variant\+: without using any helpers to work with data 
\begin{DoxyCode}
uint8\_t g\_buffer[64];

\textcolor{keywordtype}{void} onReceiveData(uint8\_t *buffer, \textcolor{keywordtype}{int} len)
\{
    \textcolor{comment}{/* Parse your received data from remote side here. */}
\}

Tiny::ProtoHd proto(g\_buffer, \textcolor{keyword}{sizeof}(g\_buffer), onReceiveData);

\textcolor{keywordtype}{void} loop()
\{
    \textcolor{keywordflow}{if} (needToSend)
    \{
        \textcolor{comment}{/* define buffer, you want to use */}
        uint8\_t buffer[16];

        \textcolor{comment}{/* Prepare data you want to send here */}
        buffer[0] = 10;
        buffer[1] = 20;

        \textcolor{comment}{/* Send 2 bytes to remote side */}
        proto.write( buffer, 2 );
    \}
    proto.run();
    ...
\}
\end{DoxyCode}
\hypertarget{arduino_arduino_tiny_hd_send_receive2}{}\subsubsection{Variant 2}\label{arduino_arduino_tiny_hd_send_receive2}
Second variant\+: with using special helper to pack the data being sent 
\begin{DoxyCode}
uint8\_t g\_buffer[64];

\textcolor{keywordtype}{void} onReceiveData(uint8\_t *buffer, \textcolor{keywordtype}{int} len)
\{
    \textcolor{comment}{/* Parse your received data here. */}
    \textcolor{comment}{/* For example, read sent ealier in example above "message" */}
    Tiny::Packet packet(buffer, len);
    \textcolor{keywordtype}{char} * str = packet.getString();
\}

Tiny::ProtoHd proto(g\_buffer, \textcolor{keyword}{sizeof}(g\_buffer), onReceiveData);

\textcolor{keywordtype}{void} loop()
\{
    \textcolor{keywordflow}{if} (needToSend)
    \{
        \textcolor{comment}{/* define buffer, you want to use */}
        uint8\_t buffer[16];
        \textcolor{comment}{/* Create helper object to simplify packing of data being sent */}
        Tiny::Packet packet(buffer, \textcolor{keyword}{sizeof}(buffer));

        \textcolor{comment}{/* Pack data you want to send here */}
        packet.put( \textcolor{stringliteral}{"message"} );

        \textcolor{comment}{/* Send message to remote side */}
        proto.write( packet );
    \}
    proto.run();
    ...
\}
\end{DoxyCode}
\hypertarget{arduino_arduino_tiny_hd_close}{}\subsection{Stopping communication}\label{arduino_arduino_tiny_hd_close}

\begin{DoxyCode}
Tiny::ProtoHd proto(g\_buffer, \textcolor{keyword}{sizeof}(g\_buffer), onReceiveData);

\textcolor{keywordtype}{void} loop()
\{
    ...
    \textcolor{keywordflow}{if} ( needToStop )
    \{
        proto.end();
    \}
\}
\end{DoxyCode}
 